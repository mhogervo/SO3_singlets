\documentclass[10pt]{article}

\pdfoutput=1 
\def\baselinestretch{1.06} 

%\usepackage[math]{iwona}
\usepackage[T1]{fontenc}

\usepackage{amsmath,amssymb,amsfonts,mathrsfs,amsthm}
%\usepackage{cmbright}
\usepackage[inline]{showlabels} 
\usepackage[]{graphicx}
\usepackage{xcolor,dsfont}
\definecolor{darkblue}{rgb}{0.1,0.1,.7}
\usepackage[colorlinks, linkcolor=darkblue, citecolor=darkblue, urlcolor=darkblue, linktocpage]{hyperref} 
\usepackage[square, comma, compress,numbers]{natbib}
\usepackage{geometry}
\geometry{letterpaper,tmargin=3.0cm,bmargin=3.0cm,lmargin=3.0cm,rmargin=3.0cm}
\usepackage[margin=10pt,font=small,labelfont=bf]{caption}
%\numberwithin{equation}{section}

\usepackage{youngtab}

%%%%%%%%%%%%% specific commands

\theoremstyle{plain}
\newtheorem*{lem}{Lemma}
\newtheorem{lemnum}{Lemma}[section]
\newtheorem*{fact}{Fact}
\theoremstyle{remark}
\newtheorem*{rem}{Remark}

%\def\zcc{\mathbb{Z}_2^{\mathcal{C}}}
\def\zb{\bar{z}}
\def\Phib{\overline{\Phi}}
\def\Wb{\overline{W}}
\def\zcc{\mathcal{C}}
\def\vol{\text{vol}(S^{d-1})}
\def\bdelta{\widetilde{\boldsymbol{a}}}
\def\bla{\boldsymbol{\lambda}}
\def\bxi{\boldsymbol{\xi}}
\def\Ricci{\mca{R}}
\def\sU{\mathrm{U}}
\def\ssU{\mathrm{SU}}
\def\NN{\mathrm{N}}
\def\SS{\mathrm{S}}
\def\be{\mathbf{e}}
\def\bm{\mathbf{m}}
\def\lmax{{\l_\text{max}}}
%% \def\vol{\text{vol}}
\def\qaq{\quad \text{and} \quad}
\def\qor{\quad \text{or} \quad}
\def\dag{\dagger}
\def\phs{{\phantom{*}}}
\def\phd{{\phantom{\dagger}}}
\def\wh{\widehat}
\def\vphi{\varphi}
\def\wt{\widetilde}
\def\bzeta{\boldsymbol{\zeta}}
\def\lra{\leftrightarrow}
\def\La{\Lambda}
\def\VV{\mathcal{V}}
\def\SO{\mathrm{SO}}
\def\Sd{\mathrm{S}_d}

\newcommand{\sset}[2]{\left\{ \, #1 \mid #2 \, \right\}}
\newcommand{\NO}[1]{{:\!#1\!:}}
\newcommand{\norm}[1]{\lvert #1 \rvert}
\newcommand{\ud}[2]{^{#1}_{\phantom{#1}#2}}
\newcommand{\du}[2]{_{#1}^{\phantom{#1}#2}}

\newcommand{\Orange}{\color [rgb]{1,.5,0}}
\newcommand{\MH}[1]{}%{\Red\bf [MH: #1]}}  

%% QM commands
\newcommand{\ceil}[1]{\left \lceil #1 \right \rceil }
\newcommand{\floor}[1]{\left \lfloor #1 \right \rfloor }
\newcommand{\braket}[3]{\langle #1|#2|#3 \rangle}
\newcommand{\brakket}[2]{\langle #1|#2\rangle}
\newcommand{\ket}[1]{|#1\rangle}
\newcommand{\bra}[1]{\langle #1|}
\newcommand{\expec}[1]{\langle #1 \rangle}

\def\dps{\displaystyle}
\def\ldef{\mathrel{\mathop:}=}
\def\rdef{=\mathrel{\mathop:}}
\newcommand{\limu}[1]{\mathrel{\mathop{\sim}\limits_{\scriptstyle{#1}}}}

%% structural TeX commands
\def\fns{\footnotesize}
\newcommand{\reef}[1]{(\ref{#1})}
\def\beq{\begin{equation}} 
\def\eeq{\end{equation}}
\def\nn{\nonumber} 
\def\bsub{\begin{subequations}}
\def\esub{\end{subequations}}

%% math styles
\def\mbb{\mathbb}
\def\mbf{\mathbf}
\def\mca{\mathcal}
\def\mfr{\mathfrak}
\def\mrm{\mathrm}
\def\msc{\mathscr}
\def\mtt{\mathtt}
\def\msf{\mathsf}

%% math symbols
% \def\vee{\, \mathrm{v}\, }
\def\th{\tfrac{1}{2}}
\def\half{\frac{1}{2}}
\def\lab{\bar{\lambda}}
\def\pd{\partial}
\def\a{\alpha}
\def\b{\beta}
\def\g{\gamma}
\def\dd{\delta}
\def\ka{\kappa}
\def\la{\lambda}
\def\ga{\gamma}
\def\ze{\zeta}
\def\DD{\Delta}
\def\Oo{\mathcal{O}}
\def\sO{\mathrm{O}}
\def\l{\ell} 
\def\eps{\epsilon}
\def\bz{\boldsymbol{\zeta}}
\def\vareps{\varepsilon}
\def\bn{\mathbf{n}}
\def\unit{\mathds{1}} % needs dsfont package

\def\ddn{\mathsf{d}_N}
\def\OON{\mathrm{O}(N)}
\def\ba{\boldsymbol{a}}
\def\bla{\boldsymbol{\lambda}}
\newcommand{\threej}[6]{ \begin{pmatrix} #1 & #2 & #3 \\ #4 & #5 & #6 \end{pmatrix}}

\def\Red{\color [rgb]{0.9,0.1,0.1}}
\def\Green{\color [rgb]{0.3,0.5,0.2}}
\def\Blue{\color [rgb]{0.3,0.5,0.8}}
\def\dim{\text{dim}}

\setlength{\parskip}{0.1in}
\hyphenpenalty=1000

\begin{document}

\title{$SO(3)$ singlets}
\author{MJH}
\date\today

\maketitle

In Hamiltonian truncation (or in other physics settings), we often work with tensor products of $SO(3)$ representations:
\beq
\label{eq:bas}
\ket{\ell_1,m_1;\ldots;\ell_N,m_N} = \ket{\ell_1,m_1} \otimes \dotsm \otimes \ket{\ell_N,m_N}.
\eeq
We have in mind that such states correspond to $N$ \emph{different} particles, so e.g.\@ the state $\ket{0,0;1,-1}$ is different from $\ket{1,-1;0,0}$. It is often useful or even crucial to organize such states into irreps of $SO(3)$. That is to say that we want to construct tensors $T^{(\la)}(m_1,\ldots,m_N)$ such that
\beq
\ket{L,j}\!\rangle \ldef \sum_{m_i} T_L^{(j)}(m_1,\ldots,m_N) \ket{\l_1,m_1;\ldots,\l_N,m_N}
\eeq
transforms as the lowest-weight state of a spin-$L$ multiplet.\footnote{Notice that rotations only act within tensor products $[\ell_1] \otimes \dotsm \otimes [\ell_N]$. Hence we can consider the spins $(\ell_1,\ldots,\ell_N)$ to be fixed.}
Recall that acting on a single-particle state $\ket{\l,m}$ we have
\beq
J_{-} \ket{\ell,m} = \gamma_{\ell,m} \ket{\ell,m-1}
\qaq
J_z \ket{\ell,m} = m \ket{\ell,m}
\eeq
with $\gamma_{\l,m} = \sqrt{(\ell+m)(\ell-m+1)}$. Hence the tensor $T^{(j)}_L$ should obey
\bsub
\label{eq:imp}
\begin{align}
&(m_1 + \ldots + m_N + L) \, T^{(j)}_L(m_i) = 0, \label{eq:ia}\\
&\ga_{\l_1,m_1+1} \, T^{(j)}(m_1+1,m_2,\ldots,m_N) + \ldots + \ga_{\l_N,m_N+1}\, T^{(j)}_L(m_1,\ldots,m_{N-1},m_N+1) = 0. \label{eq:ib}
\end{align}
\esub
The $j$ index labels different tensors, which can be normalized such that
\beq
\sum_{m_i} T^{(j)}(m_i) T^{(j')}(m_i) = \delta_{j,j'}.
\eeq
As a matter of principle these equations can be solved (they form a linear system), but this can be a rather long procedure. Instead we can form solutions constructively.

For $N=2$, it is well-understood how to organize basis states of the form~\reef{eq:bas} into irreps, using the Clebsch-Gordan coefficients. If $\ell_1,\ell_2$ are given, then the respective two-particle states fall into irreps of the following spins:
\beq
[\ell_1] \otimes [\ell_2] = \sum_{j = |\ell_1 - \ell_2|}^{\ell_1 + \ell_2} [j]
\eeq
If $[L]$ is included in this tensor product, then
\beq
\ket{L}\!\rangle \ldef \sum_{m_1,m_2} \brakket{\ell_1,m_1;\ell_2,m_2}{L,-L} \, \ket{\ell_1,m_1;\ell_2,m_2},
\eeq
is the desired lowest-weight state. Recall that the CG coefficients by construction obey
\bsub
\begin{align}
(m_1 + m_2 - M) \brakket{\l_1,m_1;\l_2,m_2}{L,M} &= 0,\label{eq:cga}\\
  \ga_{L,M} \brakket{\l_1,m_1;\l_2,m_2}{L,M-1} &= \ga_{\l_1,m_1+1}\brakket{\l_1,m_1+1;\l_2,m_2}{L,M} \nn \\
  &\qquad\qquad+ \ga_{\l_2,m_2+1}\brakket{\l_1,m_1;\l_2,m_2+1}{L,M}\label{eq:cgb}.
\end{align}
\esub
When $M = -L$ the LHS of the last equation vanishes, as it should. 


For $N\geq 3$ we can concatenate CG coefficients. For example, fix external spins $(\l_1,\l_2,\l_3)$ and let
\beq
T^{(j)}_L(m_i) \ldef \sum_{\mu = -j}^j \brakket{\l_1,m_1;\l_2,m_2}{j,\mu} \brakket{j,\mu;\l_3,m_3}{L,-L}.
\eeq
This tensor is non-zero provided that $\{\l_1,\l_2,j\}$ and $\{j,\l_3,L\}$ both obey the triangle inequality. We should check that the equations~\reef{eq:imp} are satisfied.
Indeed
\beq
(m_1 + m_2)  T^{(j)}_L = \sum_\mu \mu  \brakket{\l_1,m_1;\l_2,m_2}{j,\mu} \brakket{j,\mu;\l_3,m_3}{L,-L} = -(m_3 + L) T^{(j)}_L
\eeq
after using Eq.~\reef{eq:cga} twice, hence~\reef{eq:ia} is satisfied. Likewise
\begin{align*}
  \ga_{\l_1,m_1+1} \, T^{(j)}_L(m_1+1,m_2,m_3) + (1 \lra 2) &= \sum_\mu \ga_{j,\mu+1} \brakket{\l_1,m_1;\l_2,m_2}{j,\mu} \brakket{j,\mu+1;\l_3,m_3}{L,-L}\\
  &=- \ga_{\l_3,m_3+1}  \sum_\mu  \brakket{\l_1,m_1;\l_2,m_2}{j,\mu} \brakket{j,\mu;\l_3,m_3+1}{L,-L}\\
  &= - \ga_{\l_3,m_3+1} T^{(j)}_L(m_1,m_2,m_3+1).
\end{align*}
We applied~\reef{eq:cgb} twice, and in the first line we shifted the summation index $\mu \mapsto \mu + 1$. Hence~\reef{eq:ib} is satisfied as well. Finally, we can check that the tensors with different values of $j$ are different:
\begin{multline}
  \sum_{m_i} T^{(j)}_L(m_i) T^{(j')}_L(m_i) = \sum_{\mu,\mu'} \sum_{m_1,m_2} \brakket{\l_1,m_1;\l_2,m_2}{j,\mu}\brakket{\l_1,m_1;\l_2,m_2}{j',\mu'}  \\
  \times \sum_{m_3} \brakket{j,\mu;\l_3,m_3}{L,-L} \brakket{j',\mu';\l_3,m_3}{L,-L}.
\end{multline}
By orthogonality of the CG coefficients, the sum over $m_1,m_2$ evalues to $\delta_{j,j'} \delta_{\mu,\mu'}$ whence
\beq
\sum_{m_i} T^{(j)}_L(m_i) T^{(j')}_L(m_i) = \delta_{j,j'} \sum_{\mu,m_3}\brakket{j,\mu;\l_3,m_3}{L,-L} \brakket{j,\mu;\l_3,m_3}{L,-L} = \delta_{j,j'}.
\eeq

For $N \geq 4$ the same pattern applies. In complete generality we have
\begin{multline}
  T^{(j_1,\ldots,j_{N-2})}_L(m_1,\ldots,m_N) \ldef \sum_{\mu_i} \brakket{\l_1,m_1;\l_2,m_2}{j_1,\mu_1} \prod_{k=1}^{N-3} \brakket{j_k,\mu_k;\l_{k+2},m_{k+2}}{j_{k+1},\mu_{k+1}} \\
  \times \brakket{j_{N-2},\mu_{N-2};\l_N,m_N}{L,-L}.
\end{multline}
which is non-vanishing if $\{\l_1,\l_2,j_1\}$, $\{j_{N-2},\l_N,L\}$ and all $\{j_k,\l_{k+2},j_{k+1}\}$ all obey the triangle inequality. 



  { 
\bibliographystyle{utphys}
\bibliography{biblio}
  }
  
\end{document}



